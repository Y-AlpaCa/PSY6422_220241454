% Options for packages loaded elsewhere
\PassOptionsToPackage{unicode}{hyperref}
\PassOptionsToPackage{hyphens}{url}
%
\documentclass[
]{article}
\usepackage{amsmath,amssymb}
\usepackage{lmodern}
\usepackage{iftex}
\ifPDFTeX
  \usepackage[T1]{fontenc}
  \usepackage[utf8]{inputenc}
  \usepackage{textcomp} % provide euro and other symbols
\else % if luatex or xetex
  \usepackage{unicode-math}
  \defaultfontfeatures{Scale=MatchLowercase}
  \defaultfontfeatures[\rmfamily]{Ligatures=TeX,Scale=1}
\fi
% Use upquote if available, for straight quotes in verbatim environments
\IfFileExists{upquote.sty}{\usepackage{upquote}}{}
\IfFileExists{microtype.sty}{% use microtype if available
  \usepackage[]{microtype}
  \UseMicrotypeSet[protrusion]{basicmath} % disable protrusion for tt fonts
}{}
\makeatletter
\@ifundefined{KOMAClassName}{% if non-KOMA class
  \IfFileExists{parskip.sty}{%
    \usepackage{parskip}
  }{% else
    \setlength{\parindent}{0pt}
    \setlength{\parskip}{6pt plus 2pt minus 1pt}}
}{% if KOMA class
  \KOMAoptions{parskip=half}}
\makeatother
\usepackage{xcolor}
\usepackage[margin=1in]{geometry}
\usepackage{color}
\usepackage{fancyvrb}
\newcommand{\VerbBar}{|}
\newcommand{\VERB}{\Verb[commandchars=\\\{\}]}
\DefineVerbatimEnvironment{Highlighting}{Verbatim}{commandchars=\\\{\}}
% Add ',fontsize=\small' for more characters per line
\usepackage{framed}
\definecolor{shadecolor}{RGB}{248,248,248}
\newenvironment{Shaded}{\begin{snugshade}}{\end{snugshade}}
\newcommand{\AlertTok}[1]{\textcolor[rgb]{0.94,0.16,0.16}{#1}}
\newcommand{\AnnotationTok}[1]{\textcolor[rgb]{0.56,0.35,0.01}{\textbf{\textit{#1}}}}
\newcommand{\AttributeTok}[1]{\textcolor[rgb]{0.77,0.63,0.00}{#1}}
\newcommand{\BaseNTok}[1]{\textcolor[rgb]{0.00,0.00,0.81}{#1}}
\newcommand{\BuiltInTok}[1]{#1}
\newcommand{\CharTok}[1]{\textcolor[rgb]{0.31,0.60,0.02}{#1}}
\newcommand{\CommentTok}[1]{\textcolor[rgb]{0.56,0.35,0.01}{\textit{#1}}}
\newcommand{\CommentVarTok}[1]{\textcolor[rgb]{0.56,0.35,0.01}{\textbf{\textit{#1}}}}
\newcommand{\ConstantTok}[1]{\textcolor[rgb]{0.00,0.00,0.00}{#1}}
\newcommand{\ControlFlowTok}[1]{\textcolor[rgb]{0.13,0.29,0.53}{\textbf{#1}}}
\newcommand{\DataTypeTok}[1]{\textcolor[rgb]{0.13,0.29,0.53}{#1}}
\newcommand{\DecValTok}[1]{\textcolor[rgb]{0.00,0.00,0.81}{#1}}
\newcommand{\DocumentationTok}[1]{\textcolor[rgb]{0.56,0.35,0.01}{\textbf{\textit{#1}}}}
\newcommand{\ErrorTok}[1]{\textcolor[rgb]{0.64,0.00,0.00}{\textbf{#1}}}
\newcommand{\ExtensionTok}[1]{#1}
\newcommand{\FloatTok}[1]{\textcolor[rgb]{0.00,0.00,0.81}{#1}}
\newcommand{\FunctionTok}[1]{\textcolor[rgb]{0.00,0.00,0.00}{#1}}
\newcommand{\ImportTok}[1]{#1}
\newcommand{\InformationTok}[1]{\textcolor[rgb]{0.56,0.35,0.01}{\textbf{\textit{#1}}}}
\newcommand{\KeywordTok}[1]{\textcolor[rgb]{0.13,0.29,0.53}{\textbf{#1}}}
\newcommand{\NormalTok}[1]{#1}
\newcommand{\OperatorTok}[1]{\textcolor[rgb]{0.81,0.36,0.00}{\textbf{#1}}}
\newcommand{\OtherTok}[1]{\textcolor[rgb]{0.56,0.35,0.01}{#1}}
\newcommand{\PreprocessorTok}[1]{\textcolor[rgb]{0.56,0.35,0.01}{\textit{#1}}}
\newcommand{\RegionMarkerTok}[1]{#1}
\newcommand{\SpecialCharTok}[1]{\textcolor[rgb]{0.00,0.00,0.00}{#1}}
\newcommand{\SpecialStringTok}[1]{\textcolor[rgb]{0.31,0.60,0.02}{#1}}
\newcommand{\StringTok}[1]{\textcolor[rgb]{0.31,0.60,0.02}{#1}}
\newcommand{\VariableTok}[1]{\textcolor[rgb]{0.00,0.00,0.00}{#1}}
\newcommand{\VerbatimStringTok}[1]{\textcolor[rgb]{0.31,0.60,0.02}{#1}}
\newcommand{\WarningTok}[1]{\textcolor[rgb]{0.56,0.35,0.01}{\textbf{\textit{#1}}}}
\usepackage{graphicx}
\makeatletter
\def\maxwidth{\ifdim\Gin@nat@width>\linewidth\linewidth\else\Gin@nat@width\fi}
\def\maxheight{\ifdim\Gin@nat@height>\textheight\textheight\else\Gin@nat@height\fi}
\makeatother
% Scale images if necessary, so that they will not overflow the page
% margins by default, and it is still possible to overwrite the defaults
% using explicit options in \includegraphics[width, height, ...]{}
\setkeys{Gin}{width=\maxwidth,height=\maxheight,keepaspectratio}
% Set default figure placement to htbp
\makeatletter
\def\fps@figure{htbp}
\makeatother
\setlength{\emergencystretch}{3em} % prevent overfull lines
\providecommand{\tightlist}{%
  \setlength{\itemsep}{0pt}\setlength{\parskip}{0pt}}
\setcounter{secnumdepth}{-\maxdimen} % remove section numbering
\ifLuaTeX
  \usepackage{selnolig}  % disable illegal ligatures
\fi
\IfFileExists{bookmark.sty}{\usepackage{bookmark}}{\usepackage{hyperref}}
\IfFileExists{xurl.sty}{\usepackage{xurl}}{} % add URL line breaks if available
\urlstyle{same} % disable monospaced font for URLs
\hypersetup{
  pdftitle={Number of Recorded Natural Disasters},
  pdfauthor={220241454},
  hidelinks,
  pdfcreator={LaTeX via pandoc}}

\title{Number of Recorded Natural Disasters}
\author{220241454}
\date{2023-05}

\begin{document}
\maketitle

\hypertarget{change-in-number-of-recorded-natural-disasters-1900-to-2022}{%
\section{Change in Number of Recorded Natural Disasters, 1900 to
2022}\label{change-in-number-of-recorded-natural-disasters-1900-to-2022}}

\hypertarget{research-question}{%
\subsubsection{Research Question}\label{research-question}}

This project looks into the how the number of natural disasters varied
in the last 100 years. This could allow interpretations of why and how
the numbers changed as the way they did.\\
This project makes a plot of number of recorded natural disasters from
1900 to 2022.

\hypertarget{data-origin}{%
\subsubsection{Data Origin}\label{data-origin}}

Data first published by \href{https://emdat.be/}{EM-DAT, CRED /
UCLouvain, Brussels, Belgium}, 2022-11-27. Data was downloaded from
\href{https://ourworldindata.org/grapher/number-of-natural-disaster-events?time=earliest..latest}{Our
World in Data}. This data includes all categories classified as
``natural disasters'', which includes drought, earthquakes, extreme
temperatures, extreme weather, floods, fogs, glacial lake outbursts,
landslide, dry mass movements, volcanic activity, and wildfires.

\begin{Shaded}
\begin{Highlighting}[]
\CommentTok{\#libraries}

\FunctionTok{library}\NormalTok{(tidyverse)}
\end{Highlighting}
\end{Shaded}

\begin{verbatim}
## -- Attaching core tidyverse packages ------------------------ tidyverse 2.0.0 --
## v dplyr     1.1.1     v readr     2.1.4
## v forcats   1.0.0     v stringr   1.5.0
## v ggplot2   3.4.1     v tibble    3.2.1
## v lubridate 1.9.2     v tidyr     1.3.0
## v purrr     1.0.1     
## -- Conflicts ------------------------------------------ tidyverse_conflicts() --
## x dplyr::filter() masks stats::filter()
## x dplyr::lag()    masks stats::lag()
## i Use the ]8;;http://conflicted.r-lib.org/conflicted package]8;; to force all conflicts to become errors
\end{verbatim}

\begin{Shaded}
\begin{Highlighting}[]
\FunctionTok{library}\NormalTok{(magick)}
\end{Highlighting}
\end{Shaded}

\begin{verbatim}
## Linking to ImageMagick 6.9.12.3
## Enabled features: cairo, freetype, fftw, ghostscript, heic, lcms, pango, raw, rsvg, webp
## Disabled features: fontconfig, x11
\end{verbatim}

\begin{Shaded}
\begin{Highlighting}[]
\FunctionTok{library}\NormalTok{(dplyr)}

\CommentTok{\#load data}
\NormalTok{nnd }\OtherTok{\textless{}{-}} \FunctionTok{read.csv}\NormalTok{(}\StringTok{"data/number{-}of{-}natural{-}disaster{-}events.csv"}\NormalTok{)}

\CommentTok{\# to check loading worked}
\FunctionTok{head}\NormalTok{(nnd)}
\end{Highlighting}
\end{Shaded}

\begin{verbatim}
##          Entity Code Year Number.of.reported.natural.disasters
## 1 All disasters   NA 1900                                    6
## 2 All disasters   NA 1901                                    1
## 3 All disasters   NA 1902                                   10
## 4 All disasters   NA 1903                                   12
## 5 All disasters   NA 1904                                    4
## 6 All disasters   NA 1905                                    8
\end{verbatim}

\hypertarget{data-preparation}{%
\subsection{Data preparation}\label{data-preparation}}

I selected a few disaster types and combined the rest into ``other''
because their number was small.

\begin{Shaded}
\begin{Highlighting}[]
\CommentTok{\#filter out data I want to focus on (all disasters, extreme weather, earthquake, flood, and drought)}
\CommentTok{\#rest of the data are combined as "other"}

\CommentTok{\#Group by year and entity, and sum the events}

\NormalTok{disasters\_sum }\OtherTok{\textless{}{-}}\NormalTok{nnd }\SpecialCharTok{\%\textgreater{}\%} 
  \FunctionTok{group\_by}\NormalTok{(Year, Entity) }\SpecialCharTok{\%\textgreater{}\%}
  \FunctionTok{summarise}\NormalTok{(}\AttributeTok{events\_sum =} \FunctionTok{sum}\NormalTok{(Number.of.reported.natural.disasters))}
\end{Highlighting}
\end{Shaded}

\begin{verbatim}
## `summarise()` has grouped output by 'Year'. You can override using the
## `.groups` argument.
\end{verbatim}

\begin{Shaded}
\begin{Highlighting}[]
\CommentTok{\#Sum "other" events }

\NormalTok{other }\OtherTok{\textless{}{-}}\NormalTok{ disasters\_sum }\SpecialCharTok{\%\textgreater{}\%}
  \FunctionTok{filter}\NormalTok{(Entity }\SpecialCharTok{\%in\%} \FunctionTok{c}\NormalTok{(}\StringTok{"Fog"}\NormalTok{, }\StringTok{"Glacial lake outburst"}\NormalTok{, }\StringTok{"Dry mass movement"}\NormalTok{, }\StringTok{"Volcanic activity"}\NormalTok{, }\StringTok{"Extreme temperature"}\NormalTok{, }\StringTok{"Landslide"}\NormalTok{)) }\SpecialCharTok{\%\textgreater{}\%} 
  \FunctionTok{group\_by}\NormalTok{(Year) }\SpecialCharTok{\%\textgreater{}\%}
  \FunctionTok{summarise}\NormalTok{(}\AttributeTok{events\_sum =} \FunctionTok{sum}\NormalTok{(events\_sum)) }\SpecialCharTok{\%\textgreater{}\%} 
  \FunctionTok{mutate}\NormalTok{(}\AttributeTok{Entity =} \StringTok{"Other"}\NormalTok{)}

\CommentTok{\#Combine data frames}

\NormalTok{binded }\OtherTok{\textless{}{-}} \FunctionTok{bind\_rows}\NormalTok{(disasters\_sum, other)}

\CommentTok{\#sort by yeaar}

\NormalTok{binded }\OtherTok{\textless{}{-}}\NormalTok{ binded }\SpecialCharTok{\%\textgreater{}\%} 
  \FunctionTok{arrange}\NormalTok{(Year)}

\CommentTok{\#filter}

\NormalTok{nnd\_filtered }\OtherTok{\textless{}{-}}\NormalTok{ binded }\SpecialCharTok{\%\textgreater{}\%} 
  \FunctionTok{filter}\NormalTok{(Entity }\SpecialCharTok{\%in\%} \FunctionTok{c}\NormalTok{(}\StringTok{"All disasters"}\NormalTok{,}\StringTok{"Extreme weather"}\NormalTok{,}\StringTok{"Earthquake"}\NormalTok{,}\StringTok{"Flood"}\NormalTok{,}\StringTok{"Drought"}\NormalTok{,}\StringTok{"Other"}\NormalTok{))}

\CommentTok{\#check the first few lines}

\FunctionTok{head}\NormalTok{(nnd\_filtered)}
\end{Highlighting}
\end{Shaded}

\begin{verbatim}
## # A tibble: 6 x 3
## # Groups:   Year [1]
##    Year Entity          events_sum
##   <int> <chr>                <int>
## 1  1900 All disasters            6
## 2  1900 Drought                  2
## 3  1900 Earthquake               1
## 4  1900 Extreme weather          1
## 5  1900 Flood                    1
## 6  1900 Other                    1
\end{verbatim}

\begin{Shaded}
\begin{Highlighting}[]
\CommentTok{\#save the data frame}

\FunctionTok{write.csv}\NormalTok{(nnd\_filtered, }\AttributeTok{file =} \StringTok{"processed/nnd\_filtered.csv"}\NormalTok{, }\AttributeTok{row.names=}\ConstantTok{FALSE}\NormalTok{)}
\end{Highlighting}
\end{Shaded}

\hypertarget{data-visualisation}{%
\subsubsection{Data visualisation}\label{data-visualisation}}

\begin{Shaded}
\begin{Highlighting}[]
\CommentTok{\#setting the order of legand}

\NormalTok{nnd\_filtered}\SpecialCharTok{$}\NormalTok{Entity }\OtherTok{\textless{}{-}} 
  \FunctionTok{factor}\NormalTok{(nnd\_filtered}\SpecialCharTok{$}\NormalTok{Entity, }\AttributeTok{levels =} \FunctionTok{c}\NormalTok{(}\StringTok{"All disasters"}\NormalTok{, }\StringTok{"Flood"}\NormalTok{, }\StringTok{"Extreme weather"}\NormalTok{, }\StringTok{"Earthquake"}\NormalTok{, }\StringTok{"Drought"}\NormalTok{,}\StringTok{"Other"}\NormalTok{))}

\CommentTok{\#plot a graph}

\NormalTok{bestfit }\OtherTok{\textless{}{-}} \FunctionTok{ggplot}\NormalTok{(nnd\_filtered, }\FunctionTok{aes}\NormalTok{ (}\AttributeTok{x =}\NormalTok{ Year, }\AttributeTok{y =}\NormalTok{ events\_sum, }\AttributeTok{group =}\NormalTok{ Entity)) }\SpecialCharTok{+} 
  \FunctionTok{geom\_smooth}\NormalTok{ (}\AttributeTok{method =} \StringTok{"gam"}\NormalTok{, }\AttributeTok{se =}\NormalTok{ F, }\AttributeTok{span =} \DecValTok{2}\NormalTok{, }\FunctionTok{aes}\NormalTok{ (}\AttributeTok{col =}\NormalTok{ Entity), }\AttributeTok{alpha =} \FloatTok{0.5}\NormalTok{, }\AttributeTok{size =} \FloatTok{0.5}\NormalTok{) }\SpecialCharTok{+} 
  \FunctionTok{geom\_line}\NormalTok{ (}\AttributeTok{alpha =} \FloatTok{0.8}\NormalTok{, }\AttributeTok{size =} \FloatTok{0.8}\NormalTok{, }\FunctionTok{aes}\NormalTok{ (}\AttributeTok{col =}\NormalTok{ Entity))  }\SpecialCharTok{+} 
  \FunctionTok{ggtitle}\NormalTok{(}\StringTok{"Number of Natural Disaster Events"}\NormalTok{) }\SpecialCharTok{+}
  \FunctionTok{labs}\NormalTok{(}\AttributeTok{title =} \StringTok{"Number of Recorded Natural Disasters"}\NormalTok{,}\AttributeTok{subtitle =} \StringTok{"1900 to 2022"}\NormalTok{, }
       \AttributeTok{x =} \StringTok{"Year"}\NormalTok{, }\AttributeTok{y =} \StringTok{"Number of events"}\NormalTok{, }
       \AttributeTok{fill =} \StringTok{"natural disaster types"}\NormalTok{, }\AttributeTok{color =} \StringTok{"Disaster Category"}\NormalTok{) }\SpecialCharTok{+}
  \FunctionTok{theme}\NormalTok{(}\AttributeTok{plot.title =} \FunctionTok{element\_text}\NormalTok{(}\AttributeTok{face =} \StringTok{"bold"}\NormalTok{, }\AttributeTok{hjust =} \FloatTok{0.5}\NormalTok{), }\AttributeTok{plot.subtitle =} \FunctionTok{element\_text}\NormalTok{(}\AttributeTok{hjust =} \FloatTok{0.5}\NormalTok{))}
\NormalTok{bestfit}
\end{Highlighting}
\end{Shaded}

\begin{verbatim}
## `geom_smooth()` using formula = 'y ~ s(x, bs = "cs")'
\end{verbatim}

\includegraphics{nnd_project_knit_files/figure-latex/unnamed-chunk-3-1.pdf}

It could be suggested that there is strong increase in number of
disasters recorded, specifically for disasters like flood and extreme
weathers, which could be affected by human activity.

\begin{Shaded}
\begin{Highlighting}[]
\CommentTok{\#save the image}

\FunctionTok{ggsave}\NormalTok{(}\StringTok{"nnd\_bestfit.png"}\NormalTok{, }\AttributeTok{path =} \StringTok{"outputs/"}\NormalTok{, }\AttributeTok{plot =}\NormalTok{ bestfit, }\AttributeTok{width=}\DecValTok{15}\NormalTok{, }\AttributeTok{height=}\DecValTok{10}\NormalTok{,}\AttributeTok{dpi=}\DecValTok{300}\NormalTok{)}
\end{Highlighting}
\end{Shaded}

\begin{verbatim}
## `geom_smooth()` using formula = 'y ~ s(x, bs = "cs")'
\end{verbatim}

\hypertarget{summary}{%
\subsubsection{Summary}\label{summary}}

This data could only suggest how the recording of the natural disaster
haev changed across the year. This means many variables can affect the
data, and it can be hard to make interpretations. This could be improved
by including how human activities are related to natural disasters, and
compare this data to changes in number of disasters like earthquakes,
which can not be affected by human activities.

\end{document}
